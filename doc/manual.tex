\documentclass{article}
\usepackage{hyperref}

\usepackage{natbib}

\newcommand{\cm}[1]{\begin{center}{\tt #1}\end{center}}

\title{{\tt scrm} manual}
\author{Paul R. Staab}

\begin{document}
\maketitle

\section{Blah Blah Introduction or something}

\section{Download and installation}
{\tt scrm} can be downloaded from \url{https://...}. Extract the source code by executing the following command:
\cm{tar -xf scrm-VERSION.tar.gz.}

It is fairly standard to compile {\tt scrm} on UNIX-like systems. In the directory of {\tt scrm-VERSION}, execute the following command:
\begin{verbatim}
$./bootstrap
$make
\end{verbatim}

\section{Blah Blah examples or something}
\nocite{mcvean_approximating_2005}

\bibliographystyle{plain}
\bibliography{manual}


\end{document}
