\documentclass{book}
\usepackage[a4paper,top=2.5cm,bottom=2.5cm,left=2.5cm,right=2.5cm]{geometry}
\usepackage{makeidx}
\usepackage{natbib}
\usepackage{graphicx}
\usepackage{multicol}
\usepackage{float}
\usepackage{listings}
\usepackage{color}
\usepackage{ifthen}
\usepackage[table]{xcolor}
\usepackage{textcomp}
\usepackage{alltt}
\usepackage{ifpdf}
\ifpdf
\usepackage[pdftex,
            pagebackref=true,
            colorlinks=true,
            linkcolor=blue,
            unicode
           ]{hyperref}
\else
\usepackage[ps2pdf,
            pagebackref=true,
            colorlinks=true,
            linkcolor=blue,
            unicode
           ]{hyperref}
\usepackage{pspicture}
\fi
\usepackage[utf8]{inputenc}
\usepackage{mathptmx}
\usepackage[scaled=.90]{helvet}
\usepackage{courier}
\usepackage{sectsty}
\usepackage{amssymb}
\usepackage[titles]{tocloft}
\usepackage{doxygen}
\lstset{language=C++,inputencoding=utf8,basicstyle=\footnotesize,breaklines=true,breakatwhitespace=true,tabsize=4,numbers=left }
\makeindex
\setcounter{tocdepth}{3}
\renewcommand{\footrulewidth}{0.4pt}
\renewcommand{\familydefault}{\sfdefault}
\hfuzz=15pt
\setlength{\emergencystretch}{15pt}
\hbadness=750
\tolerance=750




% \usepackage{fullpage}
\usepackage{graphics,graphicx,subfig}
\usepackage{amsmath,multicol,lscape,framed,amssymb}
\usepackage{graphicx,epic,subfig,graphicx,tikz,pifont}
%\usepackage{multirow,epic}
% \usepackage{algorithmic,algorithm}
\usepackage{hyperref,hypcap}
\usepackage{natbib}%
\usepackage{longtable}
\usepackage{textcomp}%to use \textquotesingle
% \captionsetup{type=figure}
\hypersetup{
colorlinks,%
citecolor=blue,%
filecolor=black,%
linkcolor=blue,%
urlcolor=blue,
}
\usepackage{setspace}
%\bibliographystyle{plain}

\newcommand{\cm}[1]{\begin{center}{\tt #1}\end{center}}

\title{{\tt scrm} manual}
\author{Paul R. Staab, Sha Zhu}
\date{}
\begin{document}
\maketitle


\clearemptydoublepage
\pagenumbering{roman}
\tableofcontents
\clearemptydoublepage
\pagenumbering{arabic}
\hypersetup{pageanchor=true,citecolor=blue}


\chapter{{\tt scrm} manual}
\input{../doc/manualcontent.tex}

\section{Blah Blah Introduction or something}

\section{Download and installation}
{\tt scrm} can be downloaded from \url{https://...}. Extract the source code by executing the following command:
\cm{tar -xf scrm-VERSION.tar.gz.}

It is fairly standard to compile {\tt scrm} on UNIX-like systems. In the directory of {\tt scrm-VERSION}, execute the following command:
\begin{verbatim}
$./bootstrap
$make
\end{verbatim}

\section{Basic command}
\cm{scrm nsam }


\cm{scrm nsam nreps -T [FILENAME]}


\section{Simulation for segregating sites???????}
\cm{scrm nsam nreps -t $\theta$}



\section{Recombination}


\cm{scrm nsam nreps -t $\theta$ -r $\rho$ [-nsitis NSITES] [-npop NPOP] }

\cm{scrm 6 3 -t 0.002 -r 0.00004 -npop 20000 }



\section{Options}
\begin{longtable}{lp{9cm}}
{\tt -T [FILENAME]} & Save output trees to file specified by {\tt FILENAME}. If {\tt FILENAME} is not given, trees are saved in file {\tt TREEFILE}.\\
 {\tt  -r} $\rho$ &  User define the recombination rate $\rho$, per gerneration per site.\\
 {\tt  -nsites} NSITES &  User define the sequence length NSITES. \\
  {\tt -t} $\theta$  &  User define the mutation rate THETA. \\
{\tt -npop} NPOP &  User define the population size NPOP. \\
 {\tt -seed} SEED  &  User define the random SEED. \\
{\tt -l} exact\_window\_length &  User define the length of the exact window. \\

 {\tt -log [LOGFILE]} & Save the log of the simulation to file specified by {\tt LOGFILE}. If {\tt LOGFILE} is not given, it is save at {\tt scrm.log} by default.  
\end{longtable}



\nocite{mcvean_approximating_2005}

\bibliographystyle{plain}
\bibliography{manual}


